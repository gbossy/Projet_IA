\documentclass[a4paper,11pt,french]{article}

\usepackage[utf8]{inputenc}

\usepackage{mathrsfs}
\usepackage[english]{babel}
\usepackage{mathtools} % includes amsmath
\usepackage{amssymb}
\usepackage{amsthm}
\usepackage{amscd}
\usepackage{todonotes}

\usepackage{multirow}
\usepackage{enumerate}

\usepackage{tikz}
\usepackage{framed}
\usepackage[colorlinks]{hyperref}
\newcommand\Set[2]{\{\,#1\mid#2\,\}}


\title{Rapport de Projet en Intelligence Artificielle}
\author{Julien Sahli et Gaëtan Bossy}

\newtheorem{theorem}{Theorem}

\begin{document}

\maketitle
\section{Introduction}
\section{Tâche 1}
Nous avons utilisé \emph{Pandas} pour importer les données d'entraînement en \emph{.csv}, puis nous les avons parsées afin qu'elles correspondent au template nécessaire à notre algorithme \emph{ID3}. %Profondeur de l'arbre et élaguage ?
\section{Tâche 2}
Similairement à la tâche précédente, nous avons utilisé \emph{Pandas} pour importer les données d'entraînement, puis nous les avons parsées, et enfin nous avons prédit leur label en implémentant la fonction \emph{classifie\_type} qui retournait le type prédit par l'arbre de la première tâche. 
\end{document}
